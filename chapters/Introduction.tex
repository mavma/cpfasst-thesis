% !TEX root = ../main.tex
\chapter{Introduction}
\label{chapter:introduction}

The use of multiple programming languages in the development of a project is a common occurrence, both in general software development \cite{mayer2017multi}, and in the narrower field of scientific computing. While the use of a single language precludes several potential issues during development, it is often undesirable or unfeasible, in situations such as:
\begin{itemize}
	\item When a legacy implementation is extended using a different language, rewriting the existing implementation in the new language can be impractical due to time constraints or significant differences between the libraries;
	\item When a well-validated and optimized implementation of critical functionality is available as a library written in a different language, effort may be better invested in correctly interfacing with it than in developing an equivalent in the project's language. A common example is the numerous math libraries in different languages that leverage Fortran implementations of BLAS;
	\item Even when no existing code in other language is involved, it can be advantageous to implement different components in different programming languages, in order to leverage each language's strengths. That enables, for instance, the offer of a high level interface in Python, while boosting performance by implementing computationally intensive functionality in C.
\end{itemize}

While languages such as Fortran and MATLAB are commonly associated with scientific computing, a survey \cite{prabhu2011survey} of researchers in multiple disciplines shows that other languages, such as C, C++, Python and R, have similarly widespread adoption, and a majority of scientists make use of more than one language in any given project. Additionally, widespread use of Fortran was found to be concentrated in a few fields, and largely driven by the reuse of existing Fortran code, supporting the existence of a demand for interfaces to Fortran in different languages.

Interoperability is the ability for two (or more) programming languages to be used in the implementation of different components of the same system \cite{malone2014interoperability}. The prevalence of multi-language programs has prompted several programming languages to provide native support for interoperability with a different language (most often C), and the creation of third-party mechanisms to provide common interfaces for several languages. However, even when such a mechanism is available, the presence of tightly coupled modules or complex interfaces between different languages can still pose a significant challenge, and correct implementation requires awareness of the inner workings of each language \cite{chisnall2013challenge}, most of which are transparent to the developer in single-language environments.

This thesis focuses on interoperability between C and Fortran, as applied to the development of cpfasst, a C interface for the Fortran library LibPFASST. \refChapter{chapter:pfasst} presents an overview of the Parallel Approximation in Space and Time (PFASST) method, which LibPFASST implements. \refChapter{chapter:interoperability} discusses interoperability concerns specific to C and Fortran. \refChapter{chapter:implementation} focuses on the architecture chosen for the cpfasst implementation and how it addresses interoperability concerns, as well as the verification methods identified for the interoperable interface. \refChapter{chapter:results} presents the resulting code package and synthesizes the lessons learned about design of interoperable interfaces. Finally, \refChapter{chapter:conclusion} summarizes the work done and presents suggestions for future work.